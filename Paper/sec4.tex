\section{How do we use Neural Networks in Stock Trading?}

\subsection{Buy-Hold-Sell Signalling}
We use the paper by Kimoto, \cite{kimoto} to examine this method. The idea can be visualised in figure \ref{appl}.

\begin{figure}[h]
    \centering
    \label{aapl}
    \includegraphics[scale=0.5]{aapl.PNG}
    \caption{\$AAPL Stock with ideal buy (black) and sell (red) indicators}
\end{figure}

The technique is to train a neural network to predict when it should buy or sell based on past behaviour of the stock. This can lead to 
continuing profit over a training window.  The model in \cite{kimoto} uses  a 3-layer network, with only one hidden layer. The input of the network takes
a vector curve- showing how the stock has moved in the past, the turnover, domestic interest rate, foreign exchange rate, the New York Dow-Jones average and any 
other metrics at the investors will.\\

The authors found this method to be more profitable than simply buying and holding the stock. 

\subsection{Value Prediction}
As opposed to the previous method for directly making profit by letting the model itself trade on a stock market, this method is used for 
trying to predict the price of a stock at some future time, thus allowing an investor to either buy and hold, or go short on the stock, making a profit if the stock 
price falls. \\

In one study, \cite{adebiyi}, the authors compare a neural network model to an ARIMA (Auto-Regregressive Integrated Moving Average) model for price 
prediction.  For the experiment, they ran different network configurations, with a different range of iterations for the training. The authors found
that while both gave good forecast results, but the neural network did outperform the ARIMA model overall. 

This method is suitable for informing investment decisions, however not so much for autnomously trading.
