\section{Why use Neural Networks in Stock Trading?}
Historically, the use of financial ratios, the Efficient Market Hypothesis and The Capital Asset Pricing Model were key methods for 
forecasting the price of financial assets. There are many financial ratios and characteristics, too many for a human to look at and identify underlying
patterns. \\

This is where Neural Networks come in, they can identify important information that may go undetected by the human analyst \cite{burrell}. The immense amount of financial 
data available to the modern investor means NNs can be trained very effectively, and again pick up on relatioships between financial data
that would be missed by the human investor.  \\

However, the issue here is that the networks cannot \textit{explain} why they make the decisions they do. This is fine if the model is making 
money, however, if the model starts to malfunction, it can be hard to debug it and find out why. This is why so much ``backtesting'' is necessary- 
the practice of running a model on unseen past data to see how it performs, before giving it real money.

